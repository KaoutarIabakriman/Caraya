% Rapport de projet - Caraya Premium Car Rental
% Auteurs: Abdellah Raissouni et Kaoutar Iabakriman
% Professeur: Mohamed Chrayah

\documentclass[12pt,a4paper]{report}
\usepackage[utf8]{inputenc}
\usepackage[T1]{fontenc}
\usepackage[french]{babel}
\usepackage{graphicx}
\usepackage{hyperref}
\usepackage{xcolor}
\usepackage{geometry}
\usepackage{titlesec}
\usepackage{fancyhdr}
\usepackage{tikz}
\usetikzlibrary{shapes.geometric,positioning}
\usepackage{enumitem}
\usepackage{float}
\usepackage{booktabs}
\usepackage{array}
\usepackage{tabularx}
\usepackage{caption}

% Configuration de la page
\geometry{margin=2.5cm}
\setlength{\headheight}{15pt}

% Configuration des liens
\hypersetup{
    colorlinks=true,
    linkcolor=blue,
    filecolor=magenta,
    urlcolor=blue,
}

% Configuration des titres
\titleformat{\chapter}[display]
{\normalfont\huge\bfseries}{\chaptertitlename\ \thechapter}{20pt}{\Huge}
\titlespacing*{\chapter}{0pt}{-20pt}{40pt}

% Configuration de l'en-tête et du pied de page
\pagestyle{fancy}
\fancyhf{}
\fancyhead[L]{\slshape\nouppercase{\leftmark}}
\fancyhead[R]{\thepage}
\fancyfoot[C]{\textit{Caraya - Premium Car Rental}}
\renewcommand{\headrulewidth}{0.4pt}
\renewcommand{\footrulewidth}{0.4pt}

% Début du document
\begin{document}

% Page de garde
\begin{titlepage}
    \centering
    \vspace*{1cm}
    {\includegraphics[width=0.7\textwidth]{example-image}\par}
    \vspace{1.5cm}
    {\huge\bfseries Caraya - Premium Car Rental\par}
    \vspace{1cm}
    {\Large\bfseries Rapport de Projet\par}
    \vspace{1.5cm}
    {\large Réalisé par:\par}
    \vspace{0.5cm}
    {\large\textbf{Abdellah Raissouni}\par}
    {\large\textbf{Kaoutar Iabakriman}\par}
    \vspace{1.5cm}
    {\large Encadré par:\par}
    \vspace{0.5cm}
    {\large\textbf{Pr. Mohamed Chrayah}\par}
    \vfill
    {\large\textit{Année Académique 2023-2024}\par}
\end{titlepage}

% Table des matières
\tableofcontents
\newpage

% Remerciements
\chapter*{Remerciements}
\addcontentsline{toc}{chapter}{Remerciements}

Nous tenons à exprimer notre profonde gratitude à toutes les personnes qui ont contribué à la réalisation de ce projet.

Tout d'abord, nous adressons nos sincères remerciements à notre professeur encadrant, Pr. Mohamed Chrayah, pour son soutien constant, ses conseils précieux et sa disponibilité tout au long de ce projet. Sa guidance et son expertise ont été déterminantes pour mener à bien ce travail.

Nous remercions également l'ensemble du corps professoral pour la qualité de la formation dispensée, qui nous a fourni les connaissances et compétences nécessaires pour réaliser ce projet.

Enfin, nous exprimons notre reconnaissance à nos familles et amis pour leur soutien moral et leurs encouragements qui nous ont permis de persévérer dans les moments difficiles.

\newpage

% Introduction
\chapter{Introduction}
\section{Contexte du projet}
Dans un monde où la mobilité est devenue un besoin fondamental, les services de location de voitures occupent une place importante dans le secteur des transports. Face à une clientèle de plus en plus exigeante et à la recherche d'expériences premium, nous avons développé Caraya, une application web de location de voitures de luxe.

Ce projet s'inscrit dans une démarche d'innovation technologique visant à simplifier et à améliorer l'expérience utilisateur dans le domaine de la location automobile haut de gamme. L'application Caraya se distingue par son interface élégante, ses fonctionnalités avancées et son système de gestion efficace.

\section{Objectifs du projet}
Les principaux objectifs de notre projet sont les suivants:
\begin{itemize}
    \item Développer une plateforme web intuitive et esthétique pour la location de voitures de luxe
    \item Mettre en place un système de recherche et de filtrage avancé pour les véhicules
    \item Implémenter un système de réservation sécurisé et efficace
    \item Créer des interfaces d'administration pour la gestion des véhicules, des clients et des réservations
    \item Fournir des outils d'analyse pour optimiser la gestion de la flotte et des revenus
\end{itemize}

\section{Méthodologie}
Pour mener à bien ce projet, nous avons adopté une approche de développement agile, avec des cycles courts de développement et des révisions régulières. Nous avons commencé par une phase d'analyse des besoins, suivie de la conception de l'architecture et des interfaces, puis du développement et des tests.

La stack technologique choisie comprend Flask et MongoDB pour le backend, et Next.js avec React pour le frontend, offrant ainsi une combinaison puissante pour développer une application web moderne, performante et évolutive.

\chapter{Analyse et Conception}
\section{Analyse des besoins}
\subsection{Besoins fonctionnels}
L'application Caraya doit répondre aux besoins fonctionnels suivants:
\begin{itemize}
    \item Permettre aux utilisateurs de rechercher et de filtrer les voitures disponibles
    \item Afficher les détails complets des véhicules avec galerie d'images
    \item Gérer le processus de réservation de bout en bout
    \item Administrer l'inventaire des véhicules (ajout, modification, suppression)
    \item Gérer les clients et leurs informations
    \item Suivre l'historique des locations pour chaque véhicule
    \item Générer des rapports et des analyses sur l'activité
\end{itemize}

\subsection{Besoins non fonctionnels}
Les besoins non fonctionnels identifiés sont:
\begin{itemize}
    \item Interface utilisateur intuitive et esthétique
    \item Temps de réponse rapide pour les requêtes
    \item Sécurité des données des utilisateurs
    \item Compatibilité avec différents appareils (responsive design)
    \item Évolutivité pour supporter la croissance de l'application
\end{itemize}

\section{Diagrammes UML}
\subsection{Diagramme de cas d'utilisation}
Le diagramme de cas d'utilisation ci-dessous présente les principales interactions des différents acteurs avec le système:

\begin{figure}[H]
    \centering
    \begin{tikzpicture}
        % Acteurs
        \node[draw, text width=2cm, align=center] (client) at (-5,0) {Client};
        \node[draw, text width=2cm, align=center] (manager) at (5,0) {Manager};
        \node[draw, text width=2cm, align=center] (admin) at (5,-5) {Admin};
        
        % Cas d'utilisation
        \node[draw, shape=ellipse, text width=3cm, align=center] (recherche) at (0,2) {Rechercher des voitures};
        \node[draw, shape=ellipse, text width=3cm, align=center] (reservation) at (0,0) {Réserver une voiture};
        \node[draw, shape=ellipse, text width=3cm, align=center] (gestion) at (0,-2) {Gérer les voitures};
        \node[draw, shape=ellipse, text width=3cm, align=center] (clients) at (0,-4) {Gérer les clients};
        \node[draw, shape=ellipse, text width=3cm, align=center] (analytics) at (0,-6) {Consulter les analyses};
        
        % Relations
        \draw (client) -- (recherche);
        \draw (client) -- (reservation);
        \draw (manager) -- (gestion);
        \draw (manager) -- (clients);
        \draw (admin) -- (gestion);
        \draw (admin) -- (clients);
        \draw (admin) -- (analytics);
    \end{tikzpicture}
    \caption{Diagramme de cas d'utilisation de Caraya}
\end{figure}

\subsection{Diagramme de classes}
Le diagramme de classes suivant représente les principales entités du système et leurs relations:

\begin{figure}[H]
    \centering
    \begin{tikzpicture}[node distance=4cm]
        % Classes
        \node[draw, rectangle, text width=3.5cm, align=center] (car) at (0,0) {
            \textbf{Car} \\
            \rule{\linewidth}{0.5pt} \\
            id: String \\
            brand: String \\
            model: String \\
            year: Integer \\
            price\_per\_day: Float \\
            availability\_status: String \\
            features: Array \\
            images: Array \\
            \rule{\linewidth}{0.5pt} \\
            to\_dict() \\
            to\_json() \\
            from\_dict()
        };
        
        \node[draw, rectangle, text width=3.5cm, align=center] (client) at (-5,-5) {
            \textbf{Client} \\
            \rule{\linewidth}{0.5pt} \\
            id: String \\
            full\_name: String \\
            email: String \\
            phone: String \\
            driver\_license: String \\
            \rule{\linewidth}{0.5pt} \\
            to\_dict() \\
            to\_json() \\
            from\_dict()
        };
        
        \node[draw, rectangle, text width=3.5cm, align=center] (reservation) at (5,-5) {
            \textbf{Reservation} \\
            \rule{\linewidth}{0.5pt} \\
            id: String \\
            client\_id: String \\
            car\_id: String \\
            start\_date: Date \\
            end\_date: Date \\
            total\_amount: Float \\
            status: String \\
            \rule{\linewidth}{0.5pt} \\
            to\_dict() \\
            to\_json() \\
            from\_dict()
        };
        
        % Relations
        \draw[-latex] (client) -- node[midway, above, sloped] {fait} (reservation);
        \draw[-latex] (car) -- node[midway, above, sloped] {associée à} (reservation);
    \end{tikzpicture}
    \caption{Diagramme de classes de Caraya}
\end{figure}

\subsection{Diagramme de séquence}
Le diagramme de séquence suivant illustre les interactions entre les différents composants du système lors d'une réservation de voiture:

\begin{figure}[H]
    \centering
    \begin{tikzpicture}
        % Acteurs et objets
        \node[draw] (client) at (0,0) {Client};
        \node[draw] (frontend) at (3,0) {Frontend};
        \node[draw] (backend) at (6,0) {Backend};
        \node[draw] (db) at (9,0) {Base de données};
        
        % Lignes de vie
        \draw[dashed] (client) -- (0,-8);
        \draw[dashed] (frontend) -- (3,-8);
        \draw[dashed] (backend) -- (6,-8);
        \draw[dashed] (db) -- (9,-8);
        
        % Messages
        \draw[->] (0,-1) -- (3,-1) node[midway, above] {1. Recherche voiture};
        \draw[->] (3,-1.5) -- (6,-1.5) node[midway, above] {2. GET /api/cars};
        \draw[->] (6,-2) -- (9,-2) node[midway, above] {3. Query cars};
        \draw[->] (9,-2.5) -- (6,-2.5) node[midway, above] {4. Résultats};
        \draw[->] (6,-3) -- (3,-3) node[midway, above] {5. JSON response};
        \draw[->] (3,-3.5) -- (0,-3.5) node[midway, above] {6. Affichage résultats};
        
        \draw[->] (0,-4.5) -- (3,-4.5) node[midway, above] {7. Sélection voiture};
        \draw[->] (3,-5) -- (6,-5) node[midway, above] {8. POST /api/reservations};
        \draw[->] (6,-5.5) -- (9,-5.5) node[midway, above] {9. Insert réservation};
        \draw[->] (9,-6) -- (6,-6) node[midway, above] {10. Confirmation};
        \draw[->] (6,-6.5) -- (3,-6.5) node[midway, above] {11. JSON response};
        \draw[->] (3,-7) -- (0,-7) node[midway, above] {12. Confirmation réservation};
    \end{tikzpicture}
    \caption{Diagramme de séquence pour la réservation d'une voiture}
\end{figure}

\chapter{Architecture du Système}
\section{Architecture globale}
L'architecture de Caraya suit un modèle client-serveur avec une séparation claire entre le frontend et le backend:

\begin{figure}[H]
    \centering
    \begin{tikzpicture}[node distance=2cm]
        % Composants
        \node[draw, rectangle, minimum width=3cm, minimum height=1cm] (client) {Client (Navigateur)};
        \node[draw, rectangle, minimum width=3cm, minimum height=1cm] (frontend) at (0,-2) {Frontend (Next.js/React)};
        \node[draw, rectangle, minimum width=3cm, minimum height=1cm] (backend) at (0,-4) {Backend (Flask)};
        \node[draw, rectangle, minimum width=3cm, minimum height=1cm] (db) at (0,-6) {Base de données (MongoDB)};
        
        % Connexions
        \draw[<->] (client) -- (frontend);
        \draw[<->] (frontend) -- node[midway, right] {API REST} (backend);
        \draw[<->] (backend) -- (db);
    \end{tikzpicture}
    \caption{Architecture globale de Caraya}
\end{figure}

\section{Backend}
Le backend de l'application est développé avec Flask, un framework web léger et flexible pour Python. L'architecture du backend est organisée selon les principes suivants:

\begin{itemize}
    \item \textbf{Structure modulaire}: Organisation en modules (routes, modèles, utilitaires)
    \item \textbf{API RESTful}: Endpoints bien définis pour chaque ressource
    \item \textbf{Sécurité}: Authentification basée sur JWT et contrôle d'accès par rôles
    \item \textbf{Persistance des données}: Utilisation de MongoDB pour le stockage
\end{itemize}

\section{Frontend}
Le frontend est développé avec Next.js, un framework React qui offre des fonctionnalités avancées comme le rendu côté serveur et la génération de sites statiques. L'architecture du frontend comprend:

\begin{itemize}
    \item \textbf{Composants réutilisables}: Organisation modulaire des éléments d'interface
    \item \textbf{Gestion d'état}: Utilisation des hooks React pour la gestion de l'état
    \item \textbf{Routage}: Navigation fluide entre les différentes pages
    \item \textbf{Responsive design}: Interface adaptative pour tous les appareils
    \item \textbf{Animations}: Utilisation de Framer Motion pour des transitions fluides
\end{itemize}

\chapter{Fonctionnalités Principales}
\section{Recherche et filtrage des voitures}
L'application permet aux utilisateurs de rechercher des voitures selon différents critères:
\begin{itemize}
    \item Marque et modèle
    \item Disponibilité
    \item Fourchette de prix
    \item Caractéristiques (transmission, carburant, etc.)
\end{itemize}

Les résultats de recherche sont affichés sous forme de cartes interactives avec les informations essentielles sur chaque véhicule.

\section{Détails des voitures}
La page de détails d'une voiture présente:
\begin{itemize}
    \item Galerie d'images avec carousel
    \item Spécifications techniques complètes
    \item Disponibilité et prix
    \item Fonctionnalités et équipements
    \item Formulaire de réservation
\end{itemize}

\section{Système de réservation}
Le processus de réservation comprend les étapes suivantes:
\begin{itemize}
    \item Sélection des dates de location
    \item Saisie des informations du client
    \item Vérification de la disponibilité
    \item Calcul du coût total
    \item Confirmation de la réservation
\end{itemize}

\section{Gestion administrative}
L'interface d'administration permet:
\begin{itemize}
    \item La gestion complète du parc automobile
    \item Le suivi des clients et de leurs réservations
    \item La visualisation des statistiques et des analyses
    \item La gestion des utilisateurs et des rôles
\end{itemize}

\chapter{Captures d'écran de l'Application}
Cette section présente une vue détaillée des différentes interfaces et fonctionnalités de l'application Caraya.

\section{Interface Publique}

\subsection{Page d'accueil}
\begin{figure}[H]
    \centering
    \includegraphics[width=0.9\textwidth]{screenshots/home}
    \caption{Page d'accueil avec carrousel des véhicules vedettes}
\end{figure}

\begin{figure}[H]
    \centering
    \includegraphics[width=0.9\textwidth]{screenshots/home-search}
    \caption{Section de recherche avancée sur la page d'accueil}
\end{figure}

\subsection{Catalogue des Véhicules}
\begin{figure}[H]
    \centering
    \includegraphics[width=0.9\textwidth]{screenshots/catalog}
    \caption{Vue d'ensemble du catalogue des véhicules}
\end{figure}

\begin{figure}[H]
    \centering
    \includegraphics[width=0.9\textwidth]{screenshots/catalog-filters}
    \caption{Filtres avancés du catalogue}
\end{figure}

\subsection{Détails du Véhicule}
\begin{figure}[H]
    \centering
    \includegraphics[width=0.9\textwidth]{screenshots/car-details-gallery}
    \caption{Galerie photo du véhicule}
\end{figure}

\begin{figure}[H]
    \centering
    \includegraphics[width=0.9\textwidth]{screenshots/car-details-specs}
    \caption{Spécifications techniques du véhicule}
\end{figure}

\begin{figure}[H]
    \centering
    \includegraphics[width=0.9\textwidth]{screenshots/car-details-booking}
    \caption{Formulaire de réservation}
\end{figure}

\section{Espace Client}

\subsection{Authentification}
\begin{figure}[H]
    \centering
    \includegraphics[width=0.9\textwidth]{screenshots/login}
    \caption{Page de connexion}
\end{figure}

\begin{figure}[H]
    \centering
    \includegraphics[width=0.9\textwidth]{screenshots/register}
    \caption{Page d'inscription}
\end{figure}

\subsection{Tableau de Bord Client}
\begin{figure}[H]
    \centering
    \includegraphics[width=0.9\textwidth]{screenshots/client-dashboard}
    \caption{Vue d'ensemble du compte client}
\end{figure}

\begin{figure}[H]
    \centering
    \includegraphics[width=0.9\textwidth]{screenshots/client-reservations}
    \caption{Historique des réservations}
\end{figure}

\begin{figure}[H]
    \centering
    \includegraphics[width=0.9\textwidth]{screenshots/client-profile}
    \caption{Gestion du profil client}
\end{figure}

\section{Interface Administrative}

\subsection{Tableau de Bord Administrateur}
\begin{figure}[H]
    \centering
    \includegraphics[width=0.9\textwidth]{screenshots/admin-dashboard}
    \caption{Vue d'ensemble des statistiques}
\end{figure}

\begin{figure}[H]
    \centering
    \includegraphics[width=0.9\textwidth]{screenshots/admin-analytics}
    \caption{Analyses et rapports détaillés}
\end{figure}

\subsection{Gestion des Véhicules}
\begin{figure}[H]
    \centering
    \includegraphics[width=0.9\textwidth]{screenshots/admin-cars-list}
    \caption{Liste des véhicules}
\end{figure}

\begin{figure}[H]
    \centering
    \includegraphics[width=0.9\textwidth]{screenshots/admin-car-add}
    \caption{Formulaire d'ajout de véhicule}
\end{figure}

\begin{figure}[H]
    \centering
    \includegraphics[width=0.9\textwidth]{screenshots/admin-car-edit}
    \caption{Interface de modification d'un véhicule}
\end{figure}

\subsection{Gestion des Réservations}
\begin{figure}[H]
    \centering
    \includegraphics[width=0.9\textwidth]{screenshots/admin-reservations}
    \caption{Liste des réservations}
\end{figure}

\begin{figure}[H]
    \centering
    \includegraphics[width=0.9\textwidth]{screenshots/admin-reservation-details}
    \caption{Détails d'une réservation}
\end{figure}

\begin{figure}[H]
    \centering
    \includegraphics[width=0.9\textwidth]{screenshots/admin-calendar}
    \caption{Calendrier des réservations}
\end{figure}

\subsection{Gestion des Utilisateurs}
\begin{figure}[H]
    \centering
    \includegraphics[width=0.9\textwidth]{screenshots/admin-users}
    \caption{Liste des utilisateurs}
\end{figure}

\begin{figure}[H]
    \centering
    \includegraphics[width=0.9\textwidth]{screenshots/admin-user-details}
    \caption{Détails d'un utilisateur}
\end{figure}

\subsection{Paramètres du Système}
\begin{figure}[H]
    \centering
    \includegraphics[width=0.9\textwidth]{screenshots/admin-settings}
    \caption{Configuration générale}
\end{figure}

\begin{figure}[H]
    \centering
    \includegraphics[width=0.9\textwidth]{screenshots/admin-roles}
    \caption{Gestion des rôles et permissions}
\end{figure}

\section{Interfaces Responsives}
\begin{figure}[H]
    \centering
    \includegraphics[width=0.45\textwidth]{screenshots/mobile-home}
    \includegraphics[width=0.45\textwidth]{screenshots/mobile-catalog}
    \caption{Versions mobiles de la page d'accueil et du catalogue}
\end{figure}

\begin{figure}[H]
    \centering
    \includegraphics[width=0.45\textwidth]{screenshots/tablet-dashboard}
    \includegraphics[width=0.45\textwidth]{screenshots/tablet-car-details}
    \caption{Versions tablette du tableau de bord et des détails véhicule}
\end{figure}

\chapter{Conclusion}
\section{Bilan du projet}
Le développement de Caraya a permis de créer une application web complète pour la location de voitures de luxe. Les objectifs initiaux ont été atteints, avec la mise en place d'une interface utilisateur élégante, d'un système de réservation efficace et d'outils de gestion pour les administrateurs.

L'utilisation de technologies modernes comme Flask, MongoDB, Next.js et React a permis de développer une application performante, évolutive et facile à maintenir.

\section{Difficultés rencontrées}
Parmi les défis rencontrés durant le développement, nous pouvons citer:
\begin{itemize}
    \item La gestion des dates et des disponibilités pour éviter les conflits de réservation
    \item L'implémentation d'un système d'authentification sécurisé
    \item L'optimisation des performances pour le chargement des images
    \item La création d'une interface responsive adaptée à tous les appareils
\end{itemize}

\section{Perspectives d'évolution}
Pour les futures versions de l'application, plusieurs améliorations sont envisagées:
\begin{itemize}
    \item Intégration d'un système de paiement en ligne
    \item Mise en place d'un système de notation et d'avis clients
    \item Développement d'une application mobile native
    \item Ajout de fonctionnalités de géolocalisation pour la livraison des véhicules
    \item Implémentation d'un système de recommandation basé sur l'intelligence artificielle
\end{itemize}

\section{Compétences acquises}
Ce projet nous a permis de développer plusieurs compétences:
\begin{itemize}
    \item Maîtrise des frameworks web modernes (Flask, Next.js)
    \item Conception et implémentation d'API RESTful
    \item Utilisation de bases de données NoSQL (MongoDB)
    \item Développement d'interfaces utilisateur avec React et Tailwind CSS
    \item Gestion de projet et travail en équipe
\end{itemize}

\end{document} 